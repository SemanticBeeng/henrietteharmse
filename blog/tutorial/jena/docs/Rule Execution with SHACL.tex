\documentclass{amsart}
%\documentclass[a4paper,10pt]{scrartcl}

\usepackage[utf8x]{inputenc}
\usepackage[british]{babel}
%\usepackage[a4paper, inner=0.5cm, outer=0.5cm, top=1cm,
%bottom=1.5cm, bindingoffset=1cm]{geometry}
\usepackage{amsmath}
\usepackage{amssymb, latexsym}
\usepackage{longtable}
\usepackage[table]{xcolor}
\usepackage{textcomp} 
\usepackage{stmaryrd}
\usepackage{graphicx}
\usepackage{enumitem}
\usepackage{yfonts}
\usepackage{algpseudocode}
\usepackage{algorithm}
\usepackage{hyperref}
\usepackage{MnSymbol}
\usepackage{listings}
\usepackage{fancyvrb}

\setlist[enumerate]{label*=\arabic*.}
\newtheorem{theorem}{Theorem}[section]
\newtheorem{example}{Example}[section]
\newtheorem{definition}{Definition}[section]
\newtheorem{proposition}{Proposition}[section]
\newtheorem{notation}{Notation}[section]

\renewcommand{\algorithmicrequire}{\textbf{Input:}}
\renewcommand{\algorithmicensure}{\textbf{Output:}}

\title{Rule Execution with SHACL}
\author{Henriette Harmse}
\date{\today}

\pdfinfo{%
  /Title    (Rule Execution SHACL)
  /Author   (Henriette Harmse)
  /Creator  ()
  /Producer ()
  /Subject  (DL)
  /Keywords ()
}

\begin{document}
  \maketitle
  
  In my previous post, ``Using Jena and SHACL to validate RDF Data'', I have looked at how RDF data can be validated using SHACL. A closely related  concern to that of constraints checking, is rule execution, for which SHACL also can be used.

  
  \section{A SHACL Rule Example}
  We will again use an example from the SHACL specification. Assume we have the a file \texttt{rectangles.ttl} that contains the following data:
  
  \begin{small}
  \begin{Verbatim}
@prefix ex: <http://example.com/ns#> .
@prefix rdf: <http://www.w3.org/1999/02/22-rdf-syntax-ns#> .
@prefix rdfs: <http://www.w3.org/2000/01/rdf-schema#> .

ex:InvalidRectangle
	a ex:Rectangle .

ex:NonSquareRectangle
	a ex:Rectangle ;
	ex:height 2 ;
	ex:width 3 .
	
ex:SquareRectangle
	a ex:Rectangle ;
	ex:height 4 ;
	ex:width 4 . 
  \end{Verbatim}
  \end{small}
  
\vspace{5mm}

  Assuming we want to infer that when the height and width of a rectangle are equal, the rectangle represents a square, the following SHACL rule specification can be used (which we will store in \texttt{rectangleRules.ttl}):
  \begin{small}
  \begin{Verbatim}  
@prefix ex: <http://example.com/ns#> .
@prefix rdf: <http://www.w3.org/1999/02/22-rdf-syntax-ns#> .
@prefix rdfs: <http://www.w3.org/2000/01/rdf-schema#> .
@prefix dash: <http://datashapes.org/dash#> .
@prefix sh: <http://www.w3.org/ns/shacl#> .
@prefix xsd: <http://www.w3.org/2001/XMLSchema#> .

ex:Rectangle
	a rdfs:Class, sh:NodeShape ;
	rdfs:label "Rectangle" ;
	sh:property [
		sh:path ex:height ;
		sh:datatype xsd:integer ;
		sh:maxCount 1 ;
		sh:minCount 1 ;
		sh:name "height" ;
	] ;
	sh:property [
		sh:path ex:width ;
		sh:datatype xsd:integer ;
		sh:maxCount 1 ;
		sh:minCount 1 ;
		sh:name "width" ;
	] ;
	sh:rule [
		a sh:TripleRule ;
		sh:subject sh:this ;
		sh:predicate rdf:type ;
		sh:object ex:Square ;
		sh:condition ex:Rectangle ;
		sh:condition [
			sh:property [
				sh:path ex:width ;
				sh:equals ex:height ;
			] ;
		] ;
	] .
  \end{Verbatim}
  \end{small}
  
  
  \section{A Code Example using Jena}
  Naturally you will need to add SHACL to your Maven pom dependencies. Then the following code will execute your SHACL rules:
  
  \begin{small}
  \begin{Verbatim}   
package org.shacl.tutorial;

import java.io.File;
import java.io.FileOutputStream;
import java.io.OutputStream;
import java.nio.file.Path;
import java.nio.file.Paths;

import org.apache.jena.rdf.model.Model;
import org.apache.jena.riot.RDFDataMgr;
import org.apache.jena.riot.RDFFormat;
import org.slf4j.Logger;
import org.slf4j.LoggerFactory;
import org.slf4j.Marker;
import org.slf4j.MarkerFactory;
import org.topbraid.shacl.rules.RuleUtil;
import org.topbraid.spin.util.JenaUtil;

public class ShaclRuleExecution {
  private static Logger logger = LoggerFactory.getLogger(ShaclValidation.class);
  // Why This Failure marker
  private static final Marker WTF_MARKER = MarkerFactory.getMarker("WTF");

  
  public static void main(String[] args) {
    try {   
      Path path = Paths.get(".").toAbsolutePath().normalize();
      String data = "file:" + path.toFile().getAbsolutePath() + 
	  "/src/main/resources/rectangles.ttl";
      String shape = "file:" + path.toFile().getAbsolutePath() + 
	  "/src/main/resources/rectangleRules.ttl";   
      
      
      Model dataModel = JenaUtil.createDefaultModel();
      dataModel.read(data);
      Model shapeModel = JenaUtil.createDefaultModel();
      shapeModel.read(shape);
      Model inferenceModel = JenaUtil.createDefaultModel();
      
      inferenceModel = RuleUtil.executeRules(dataModel, shapeModel, 
	  inferenceModel, null);
      
      String inferences = path.toFile().getAbsolutePath() + 
	  "/src/main/resources/inferences.ttl";
      File inferencesFile = new File(inferences);
      inferencesFile.createNewFile();     
      OutputStream reportOutputStream = new FileOutputStream(inferencesFile);
      
      RDFDataMgr.write(reportOutputStream, inferenceModel, RDFFormat.TTL);        
    } catch (Throwable t) {
      logger.error(WTF_MARKER, t.getMessage(), t);
    }   
  }
}
  \end{Verbatim}
  \end{small}
  
  
  \section{Running the Code}
  Running the code will cause an \texttt{inferences.ttl} file to be written out to \\\texttt{\$Project/src/main/resources/}. It contains the following output:
  \begin{small}
  \begin{Verbatim} 
@prefix owl:   <http://www.w3.org/2002/07/owl#> .
@prefix rdf:   <http://www.w3.org/1999/02/22-rdf-syntax-ns#> .
@prefix xsd:   <http://www.w3.org/2001/XMLSchema#> .
@prefix rdfs:  <http://www.w3.org/2000/01/rdf-schema#> .

<http://example.com/ns#SquareRectangle>
        a       <http://example.com/ns#Square> .
  \end{Verbatim}
  \end{small}
  Note that \texttt{ex:InvalidRectangle} has been ignore because it does not adhere to \texttt{sh:condition ex:Rectangle}, since it does not have \texttt{ex:height} and \texttt{ex:width} properties. Also, \texttt{ex:NonSquareRectangle} is a rectangle, not a square.

  \section{Conclusion}
  In this post I gave a brief overview of how SHACL can be used to implement rules on RDF data. This code example is available at 
  \href{https://github.com/henrietteharmse/henrietteharmse/tree/master/blog/tutorial/jena/source/shacl}{shacl tutorial}.
  
  
  
  
  
  \bibliographystyle{amsplain}
  \bibliography{./BibliographicDetails_v.0.1}
 
\end{document}
